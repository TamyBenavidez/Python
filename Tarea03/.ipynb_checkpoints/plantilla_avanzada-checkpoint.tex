% Plantilla definitiva LaTeX en español para WSL/Jupyter con TeX Live 2025
\documentclass[12pt]{article}

% Paquetes básicos
\usepackage[main=spanish]{babel}   % Idioma español
\usepackage[utf8]{inputenc}        % Codificación UTF-8
\usepackage[T1]{fontenc}           % Codificación de fuentes
\usepackage{lmodern}               % Fuente moderna
\usepackage{graphicx}              % Para incluir imágenes
\usepackage{amsmath, amssymb}      % Matemáticas
\usepackage{hyperref}              % Hipervínculos
\usepackage{geometry}              % Márgenes
\geometry{margin=2.5cm}
\usepackage{caption}               % Configuración de captions
\usepackage{float}                 % Posicionamiento de figuras y tablas
\usepackage{xcolor}                % Colores para listings
\usepackage{listings}              % Bloques de código

% Configuración de listings para Python
\lstset{
    language=Python,
    basicstyle=\ttfamily\small,
    keywordstyle=\color{blue},
    commentstyle=\color{gray},
    stringstyle=\color{red},
    showstringspaces=false,
    numbers=left,
    numberstyle=\tiny,
    frame=single,
    breaklines=true
}

% Información del documento
\title{Título del Documento}
\author{Tu Nombre}
\date{\today}

\begin{document}

\maketitle

\begin{abstract}
Resumen breve del documento.
\end{abstract}

\section{Introducción}
Texto de introducción en español. Puedes incluir listas, ecuaciones y referencias.

\subsection{Lista de ejemplo}
\begin{itemize}
    \item Primer elemento
    \item Segundo elemento
    \item Tercer elemento
\end{itemize}

\subsection{Ecuación de ejemplo}
\begin{equation}
E = mc^2
\end{equation}

\section{Figura de ejemplo}
\begin{figure}[H]
    \centering
    \includegraphics[width=0.6\textwidth]{Algoritmo1.png} % Cambia por tu imagen
    \caption{Ejemplo de figura}
    \label{fig:ejemplo}
\end{figure}

\section{Tabla de ejemplo}
\begin{table}[H]
    \centering
    \begin{tabular}{|c|c|c|}
        \hline
        A & B & C \\
        \hline
        1 & 2 & 3 \\
        4 & 5 & 6 \\
        \hline
    \end{tabular}
    \caption{Ejemplo de tabla}
    \label{tab:ejemplo}
\end{table}

\section{Codigo en Python}
\begin{lstlisting}
# Codigo Python de ejemplo
import numpy as np

x = np.linspace(0, 10, 100)
y = np.sin(x)

print(y)
\end{lstlisting}

\section{Conclusión}
Conclusiones finales del documento.

\end{document}
