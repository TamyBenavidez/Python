% Options for packages loaded elsewhere
% Options for packages loaded elsewhere
\PassOptionsToPackage{unicode}{hyperref}
\PassOptionsToPackage{hyphens}{url}
\PassOptionsToPackage{dvipsnames,svgnames,x11names}{xcolor}
%
\documentclass[
  english,
  letterpaper,
  DIV=11,
  numbers=noendperiod]{scrartcl}
\usepackage{xcolor}
\usepackage{amsmath,amssymb}
\setcounter{secnumdepth}{-\maxdimen} % remove section numbering
\usepackage{iftex}
\ifPDFTeX
  \usepackage[T1]{fontenc}
  \usepackage[utf8]{inputenc}
  \usepackage{textcomp} % provide euro and other symbols
\else % if luatex or xetex
  \usepackage{unicode-math} % this also loads fontspec
  \defaultfontfeatures{Scale=MatchLowercase}
  \defaultfontfeatures[\rmfamily]{Ligatures=TeX,Scale=1}
\fi
\usepackage{lmodern}
\ifPDFTeX\else
  % xetex/luatex font selection
\fi
% Use upquote if available, for straight quotes in verbatim environments
\IfFileExists{upquote.sty}{\usepackage{upquote}}{}
\IfFileExists{microtype.sty}{% use microtype if available
  \usepackage[]{microtype}
  \UseMicrotypeSet[protrusion]{basicmath} % disable protrusion for tt fonts
}{}
\makeatletter
\@ifundefined{KOMAClassName}{% if non-KOMA class
  \IfFileExists{parskip.sty}{%
    \usepackage{parskip}
  }{% else
    \setlength{\parindent}{0pt}
    \setlength{\parskip}{6pt plus 2pt minus 1pt}}
}{% if KOMA class
  \KOMAoptions{parskip=half}}
\makeatother
% Make \paragraph and \subparagraph free-standing
\makeatletter
\ifx\paragraph\undefined\else
  \let\oldparagraph\paragraph
  \renewcommand{\paragraph}{
    \@ifstar
      \xxxParagraphStar
      \xxxParagraphNoStar
  }
  \newcommand{\xxxParagraphStar}[1]{\oldparagraph*{#1}\mbox{}}
  \newcommand{\xxxParagraphNoStar}[1]{\oldparagraph{#1}\mbox{}}
\fi
\ifx\subparagraph\undefined\else
  \let\oldsubparagraph\subparagraph
  \renewcommand{\subparagraph}{
    \@ifstar
      \xxxSubParagraphStar
      \xxxSubParagraphNoStar
  }
  \newcommand{\xxxSubParagraphStar}[1]{\oldsubparagraph*{#1}\mbox{}}
  \newcommand{\xxxSubParagraphNoStar}[1]{\oldsubparagraph{#1}\mbox{}}
\fi
\makeatother

\usepackage{color}
\usepackage{fancyvrb}
\newcommand{\VerbBar}{|}
\newcommand{\VERB}{\Verb[commandchars=\\\{\}]}
\DefineVerbatimEnvironment{Highlighting}{Verbatim}{commandchars=\\\{\}}
% Add ',fontsize=\small' for more characters per line
\usepackage{framed}
\definecolor{shadecolor}{RGB}{241,243,245}
\newenvironment{Shaded}{\begin{snugshade}}{\end{snugshade}}
\newcommand{\AlertTok}[1]{\textcolor[rgb]{0.68,0.00,0.00}{#1}}
\newcommand{\AnnotationTok}[1]{\textcolor[rgb]{0.37,0.37,0.37}{#1}}
\newcommand{\AttributeTok}[1]{\textcolor[rgb]{0.40,0.45,0.13}{#1}}
\newcommand{\BaseNTok}[1]{\textcolor[rgb]{0.68,0.00,0.00}{#1}}
\newcommand{\BuiltInTok}[1]{\textcolor[rgb]{0.00,0.23,0.31}{#1}}
\newcommand{\CharTok}[1]{\textcolor[rgb]{0.13,0.47,0.30}{#1}}
\newcommand{\CommentTok}[1]{\textcolor[rgb]{0.37,0.37,0.37}{#1}}
\newcommand{\CommentVarTok}[1]{\textcolor[rgb]{0.37,0.37,0.37}{\textit{#1}}}
\newcommand{\ConstantTok}[1]{\textcolor[rgb]{0.56,0.35,0.01}{#1}}
\newcommand{\ControlFlowTok}[1]{\textcolor[rgb]{0.00,0.23,0.31}{\textbf{#1}}}
\newcommand{\DataTypeTok}[1]{\textcolor[rgb]{0.68,0.00,0.00}{#1}}
\newcommand{\DecValTok}[1]{\textcolor[rgb]{0.68,0.00,0.00}{#1}}
\newcommand{\DocumentationTok}[1]{\textcolor[rgb]{0.37,0.37,0.37}{\textit{#1}}}
\newcommand{\ErrorTok}[1]{\textcolor[rgb]{0.68,0.00,0.00}{#1}}
\newcommand{\ExtensionTok}[1]{\textcolor[rgb]{0.00,0.23,0.31}{#1}}
\newcommand{\FloatTok}[1]{\textcolor[rgb]{0.68,0.00,0.00}{#1}}
\newcommand{\FunctionTok}[1]{\textcolor[rgb]{0.28,0.35,0.67}{#1}}
\newcommand{\ImportTok}[1]{\textcolor[rgb]{0.00,0.46,0.62}{#1}}
\newcommand{\InformationTok}[1]{\textcolor[rgb]{0.37,0.37,0.37}{#1}}
\newcommand{\KeywordTok}[1]{\textcolor[rgb]{0.00,0.23,0.31}{\textbf{#1}}}
\newcommand{\NormalTok}[1]{\textcolor[rgb]{0.00,0.23,0.31}{#1}}
\newcommand{\OperatorTok}[1]{\textcolor[rgb]{0.37,0.37,0.37}{#1}}
\newcommand{\OtherTok}[1]{\textcolor[rgb]{0.00,0.23,0.31}{#1}}
\newcommand{\PreprocessorTok}[1]{\textcolor[rgb]{0.68,0.00,0.00}{#1}}
\newcommand{\RegionMarkerTok}[1]{\textcolor[rgb]{0.00,0.23,0.31}{#1}}
\newcommand{\SpecialCharTok}[1]{\textcolor[rgb]{0.37,0.37,0.37}{#1}}
\newcommand{\SpecialStringTok}[1]{\textcolor[rgb]{0.13,0.47,0.30}{#1}}
\newcommand{\StringTok}[1]{\textcolor[rgb]{0.13,0.47,0.30}{#1}}
\newcommand{\VariableTok}[1]{\textcolor[rgb]{0.07,0.07,0.07}{#1}}
\newcommand{\VerbatimStringTok}[1]{\textcolor[rgb]{0.13,0.47,0.30}{#1}}
\newcommand{\WarningTok}[1]{\textcolor[rgb]{0.37,0.37,0.37}{\textit{#1}}}

\usepackage{longtable,booktabs,array}
\usepackage{calc} % for calculating minipage widths
% Correct order of tables after \paragraph or \subparagraph
\usepackage{etoolbox}
\makeatletter
\patchcmd\longtable{\par}{\if@noskipsec\mbox{}\fi\par}{}{}
\makeatother
% Allow footnotes in longtable head/foot
\IfFileExists{footnotehyper.sty}{\usepackage{footnotehyper}}{\usepackage{footnote}}
\makesavenoteenv{longtable}
\usepackage{graphicx}
\makeatletter
\newsavebox\pandoc@box
\newcommand*\pandocbounded[1]{% scales image to fit in text height/width
  \sbox\pandoc@box{#1}%
  \Gscale@div\@tempa{\textheight}{\dimexpr\ht\pandoc@box+\dp\pandoc@box\relax}%
  \Gscale@div\@tempb{\linewidth}{\wd\pandoc@box}%
  \ifdim\@tempb\p@<\@tempa\p@\let\@tempa\@tempb\fi% select the smaller of both
  \ifdim\@tempa\p@<\p@\scalebox{\@tempa}{\usebox\pandoc@box}%
  \else\usebox{\pandoc@box}%
  \fi%
}
% Set default figure placement to htbp
\def\fps@figure{htbp}
\makeatother



\ifLuaTeX
\usepackage[bidi=basic]{babel}
\else
\usepackage[bidi=default]{babel}
\fi
% get rid of language-specific shorthands (see #6817):
\let\LanguageShortHands\languageshorthands
\def\languageshorthands#1{}
\ifLuaTeX
  \usepackage[english]{selnolig} % disable illegal ligatures
\fi


\setlength{\emergencystretch}{3em} % prevent overfull lines

\providecommand{\tightlist}{%
  \setlength{\itemsep}{0pt}\setlength{\parskip}{0pt}}



 


\KOMAoption{captions}{tableheading}
\makeatletter
\@ifpackageloaded{caption}{}{\usepackage{caption}}
\AtBeginDocument{%
\ifdefined\contentsname
  \renewcommand*\contentsname{Table of contents}
\else
  \newcommand\contentsname{Table of contents}
\fi
\ifdefined\listfigurename
  \renewcommand*\listfigurename{List of Figures}
\else
  \newcommand\listfigurename{List of Figures}
\fi
\ifdefined\listtablename
  \renewcommand*\listtablename{List of Tables}
\else
  \newcommand\listtablename{List of Tables}
\fi
\ifdefined\figurename
  \renewcommand*\figurename{Figure}
\else
  \newcommand\figurename{Figure}
\fi
\ifdefined\tablename
  \renewcommand*\tablename{Table}
\else
  \newcommand\tablename{Table}
\fi
}
\@ifpackageloaded{float}{}{\usepackage{float}}
\floatstyle{ruled}
\@ifundefined{c@chapter}{\newfloat{codelisting}{h}{lop}}{\newfloat{codelisting}{h}{lop}[chapter]}
\floatname{codelisting}{Listing}
\newcommand*\listoflistings{\listof{codelisting}{List of Listings}}
\makeatother
\makeatletter
\makeatother
\makeatletter
\@ifpackageloaded{caption}{}{\usepackage{caption}}
\@ifpackageloaded{subcaption}{}{\usepackage{subcaption}}
\makeatother
\usepackage{bookmark}
\IfFileExists{xurl.sty}{\usepackage{xurl}}{} % add URL line breaks if available
\urlstyle{same}
\hypersetup{
  pdftitle={Tarea N°3 Algoritmos y Complejidad},
  pdfauthor={Tamara Benavidez},
  pdflang={en},
  colorlinks=true,
  linkcolor={blue},
  filecolor={Maroon},
  citecolor={Blue},
  urlcolor={Blue},
  pdfcreator={LaTeX via pandoc}}


\title{Tarea N°3 Algoritmos y Complejidad}
\author{Tamara Benavidez}
\date{}
\begin{document}
\maketitle

\renewcommand*\contentsname{Tabla de Contenidos}
{
\hypersetup{linkcolor=}
\setcounter{tocdepth}{3}
\tableofcontents
}

\section{ESCUELA POLITÉCNICA
NACIONAL}\label{escuela-polituxe9cnica-nacional}

\begin{center}
\includegraphics[width=3cm,height=3cm]{logoEpn.jpg}
\end{center}

\section{TAREA N°3 Ejercicios Unidad 01
B}\label{tarea-n3-ejercicios-unidad-01-b}

\subsection{Métodos de Newton y de la
Secante}\label{muxe9todos-de-newton-y-de-la-secante}

\subsubsection{Ejercicios}\label{ejercicios}

\begin{enumerate}
\def\labelenumi{\arabic{enumi}.}
\tightlist
\item
  Utilice aritmética de corte de tres dígitos para calcular las
  siguientes sumas. Para cada parte, ¿qué método es más preciso y por
  qué?
\end{enumerate}

\begin{enumerate}
\def\labelenumi{\alph{enumi}.}
\tightlist
\item
  \(\sum_{i=1}^{10} \left ( \frac{1}{i^2} \right)\) primero por:
  \(\frac{1}{1}+ \frac{1}{4} + \cdots + \frac{1}{100}\) Y luego por:
  \(\frac{1}{100} + \frac{1}{81} + \cdots + \frac{1}{1}\)
\end{enumerate}

\begin{Shaded}
\begin{Highlighting}[]
\ImportTok{import}\NormalTok{ math}

\CommentTok{\# Función para cortar a 3 cifras significativas}
\KeywordTok{def}\NormalTok{ truncar\_3(x):}
    \ControlFlowTok{if}\NormalTok{ x }\OperatorTok{==} \DecValTok{0}\NormalTok{:}
        \ControlFlowTok{return} \FloatTok{0.0}
\NormalTok{    potencia }\OperatorTok{=}\NormalTok{ math.floor(math.log10(}\BuiltInTok{abs}\NormalTok{(x)))}
\NormalTok{    factor }\OperatorTok{=} \DecValTok{10} \OperatorTok{**}\NormalTok{ (}\DecValTok{2} \OperatorTok{{-}}\NormalTok{ potencia)}
    \ControlFlowTok{return}\NormalTok{ math.trunc(x }\OperatorTok{*}\NormalTok{ factor) }\OperatorTok{/}\NormalTok{ factor}

\CommentTok{\# Suma con aritmética de corte a 3 cifras significativas}
\KeywordTok{def}\NormalTok{ suma\_truncada(valores, mostrar}\OperatorTok{=}\VariableTok{False}\NormalTok{):}
\NormalTok{    s }\OperatorTok{=} \FloatTok{0.0}
    \ControlFlowTok{for}\NormalTok{ v }\KeywordTok{in}\NormalTok{ valores:}
\NormalTok{        s }\OperatorTok{=}\NormalTok{ truncar\_3(s }\OperatorTok{+}\NormalTok{ v)}
        \ControlFlowTok{if}\NormalTok{ mostrar:}
            \BuiltInTok{print}\NormalTok{(}\SpecialStringTok{f"Suma parcial: }\SpecialCharTok{\{}\NormalTok{s}\SpecialCharTok{\}}\SpecialStringTok{"}\NormalTok{)}
    \ControlFlowTok{return}\NormalTok{ s}

\CommentTok{\# Sumatoria de 1/i²}
\NormalTok{valores\_a }\OperatorTok{=}\NormalTok{ [truncar\_3(}\DecValTok{1}\OperatorTok{/}\NormalTok{i}\OperatorTok{**}\DecValTok{2}\NormalTok{) }\ControlFlowTok{for}\NormalTok{ i }\KeywordTok{in} \BuiltInTok{range}\NormalTok{(}\DecValTok{1}\NormalTok{, }\DecValTok{11}\NormalTok{)]}

\BuiltInTok{print}\NormalTok{(}\StringTok{"Valores truncados (1/i²):"}\NormalTok{)}
\BuiltInTok{print}\NormalTok{(valores\_a)}

\BuiltInTok{print}\NormalTok{(}\StringTok{"}\CharTok{\textbackslash{}n}\StringTok{Suma ascendente (1 a 10)"}\NormalTok{)}
\NormalTok{suma\_asc\_a }\OperatorTok{=}\NormalTok{ suma\_truncada(valores\_a, mostrar}\OperatorTok{=}\VariableTok{True}\NormalTok{)}
\BuiltInTok{print}\NormalTok{(}\StringTok{"Suma ascendente ="}\NormalTok{, suma\_asc\_a)}

\BuiltInTok{print}\NormalTok{(}\StringTok{"}\CharTok{\textbackslash{}n}\StringTok{Suma descendente (10 a 1)"}\NormalTok{)}
\NormalTok{suma\_desc\_a }\OperatorTok{=}\NormalTok{ suma\_truncada(}\BuiltInTok{list}\NormalTok{(}\BuiltInTok{reversed}\NormalTok{(valores\_a)), mostrar}\OperatorTok{=}\VariableTok{True}\NormalTok{)}
\BuiltInTok{print}\NormalTok{(}\StringTok{"Suma descendente ="}\NormalTok{, suma\_desc\_a)}

\CommentTok{\# Comparar con valor real}
\NormalTok{real\_a }\OperatorTok{=} \BuiltInTok{sum}\NormalTok{(}\DecValTok{1}\OperatorTok{/}\NormalTok{i}\OperatorTok{**}\DecValTok{2} \ControlFlowTok{for}\NormalTok{ i }\KeywordTok{in} \BuiltInTok{range}\NormalTok{(}\DecValTok{1}\NormalTok{, }\DecValTok{11}\NormalTok{))}
\BuiltInTok{print}\NormalTok{(}\SpecialStringTok{f"}\CharTok{\textbackslash{}n}\SpecialStringTok{Valor real (doble precisión): }\SpecialCharTok{\{}\NormalTok{real\_a}\SpecialCharTok{:.6f\}}\SpecialStringTok{"}\NormalTok{)}

\BuiltInTok{print}\NormalTok{(}\StringTok{"}\CharTok{\textbackslash{}n}\StringTok{Errores absolutos:"}\NormalTok{)}
\BuiltInTok{print}\NormalTok{(}\SpecialStringTok{f"Ascendente: }\SpecialCharTok{\{}\BuiltInTok{abs}\NormalTok{(real\_a }\OperatorTok{{-}}\NormalTok{ suma\_asc\_a)}\SpecialCharTok{\}}\SpecialStringTok{"}\NormalTok{)}
\BuiltInTok{print}\NormalTok{(}\SpecialStringTok{f"Descendente: }\SpecialCharTok{\{}\BuiltInTok{abs}\NormalTok{(real\_a }\OperatorTok{{-}}\NormalTok{ suma\_desc\_a)}\SpecialCharTok{\}}\SpecialStringTok{"}\NormalTok{)}
\end{Highlighting}
\end{Shaded}

\begin{verbatim}
Valores truncados (1/i²):
[1.0, 0.25, 0.111, 0.0625, 0.04, 0.0277, 0.0204, 0.0156, 0.0123, 0.01]

Suma ascendente (1 a 10)
Suma parcial: 1.0
Suma parcial: 1.25
Suma parcial: 1.36
Suma parcial: 1.42
Suma parcial: 1.46
Suma parcial: 1.48
Suma parcial: 1.5
Suma parcial: 1.51
Suma parcial: 1.52
Suma parcial: 1.53
Suma ascendente = 1.53

Suma descendente (10 a 1)
Suma parcial: 0.01
Suma parcial: 0.0223
Suma parcial: 0.0379
Suma parcial: 0.0583
Suma parcial: 0.0859
Suma parcial: 0.125
Suma parcial: 0.187
Suma parcial: 0.298
Suma parcial: 0.548
Suma parcial: 1.54
Suma descendente = 1.54

Valor real (doble precisión): 1.549768

Errores absolutos:
Ascendente: 0.019767731166540736
Descendente: 0.009767731166540727
\end{verbatim}

\subsubsection{\texorpdfstring{b.
\(\sum_{i=1}^{10} \left( \frac{1}{i^3} \right)\) primero por:
\(\frac{1}{1} + \frac{1}{8} + \frac{1}{27} + \cdots + \frac{1}{1000}\) Y
luego por:
\(\frac{1}{1000} + \frac{1}{729} + \cdots + \frac{1}{1}\)}{b. \textbackslash sum\_\{i=1\}\^{}\{10\} \textbackslash left( \textbackslash frac\{1\}\{i\^{}3\} \textbackslash right) primero por: \textbackslash frac\{1\}\{1\} + \textbackslash frac\{1\}\{8\} + \textbackslash frac\{1\}\{27\} + \textbackslash cdots + \textbackslash frac\{1\}\{1000\} Y luego por: \textbackslash frac\{1\}\{1000\} + \textbackslash frac\{1\}\{729\} + \textbackslash cdots + \textbackslash frac\{1\}\{1\}}}\label{b.-sum_i110-left-frac1i3-right-primero-por-frac11-frac18-frac127-cdots-frac11000-y-luego-por-frac11000-frac1729-cdots-frac11}

\begin{Shaded}
\begin{Highlighting}[]
\CommentTok{\# Sumatoria de 1/i³ }
\NormalTok{valores\_b }\OperatorTok{=}\NormalTok{ [truncar\_3(}\DecValTok{1}\OperatorTok{/}\NormalTok{i}\OperatorTok{**}\DecValTok{3}\NormalTok{) }\ControlFlowTok{for}\NormalTok{ i }\KeywordTok{in} \BuiltInTok{range}\NormalTok{(}\DecValTok{1}\NormalTok{, }\DecValTok{11}\NormalTok{)]}

\BuiltInTok{print}\NormalTok{(}\StringTok{"Valores truncados (1/i³):"}\NormalTok{)}
\BuiltInTok{print}\NormalTok{(valores\_b)}

\BuiltInTok{print}\NormalTok{(}\StringTok{"}\CharTok{\textbackslash{}n}\StringTok{Suma ascendente (1 a 10)"}\NormalTok{)}
\NormalTok{suma\_asc\_b }\OperatorTok{=}\NormalTok{ suma\_truncada(valores\_b, mostrar}\OperatorTok{=}\VariableTok{True}\NormalTok{)}
\BuiltInTok{print}\NormalTok{(}\StringTok{"Suma ascendente ="}\NormalTok{, suma\_asc\_b)}

\BuiltInTok{print}\NormalTok{(}\StringTok{"}\CharTok{\textbackslash{}n}\StringTok{Suma descendente (10 a 1)"}\NormalTok{)}
\NormalTok{suma\_desc\_b }\OperatorTok{=}\NormalTok{ suma\_truncada(}\BuiltInTok{list}\NormalTok{(}\BuiltInTok{reversed}\NormalTok{(valores\_b)), mostrar}\OperatorTok{=}\VariableTok{True}\NormalTok{)}
\BuiltInTok{print}\NormalTok{(}\StringTok{"Suma descendente ="}\NormalTok{, suma\_desc\_b)}

\CommentTok{\# Comparar con valor real}
\NormalTok{real\_b }\OperatorTok{=} \BuiltInTok{sum}\NormalTok{(}\DecValTok{1}\OperatorTok{/}\NormalTok{i}\OperatorTok{**}\DecValTok{3} \ControlFlowTok{for}\NormalTok{ i }\KeywordTok{in} \BuiltInTok{range}\NormalTok{(}\DecValTok{1}\NormalTok{, }\DecValTok{11}\NormalTok{))}
\BuiltInTok{print}\NormalTok{(}\SpecialStringTok{f"}\CharTok{\textbackslash{}n}\SpecialStringTok{Valor real (doble precisión): }\SpecialCharTok{\{}\NormalTok{real\_b}\SpecialCharTok{:.6f\}}\SpecialStringTok{"}\NormalTok{)}

\BuiltInTok{print}\NormalTok{(}\StringTok{"}\CharTok{\textbackslash{}n}\StringTok{Errores absolutos:"}\NormalTok{)}
\BuiltInTok{print}\NormalTok{(}\SpecialStringTok{f"Ascendente: }\SpecialCharTok{\{}\BuiltInTok{abs}\NormalTok{(real\_b }\OperatorTok{{-}}\NormalTok{ suma\_asc\_b)}\SpecialCharTok{\}}\SpecialStringTok{"}\NormalTok{)}
\BuiltInTok{print}\NormalTok{(}\SpecialStringTok{f"Descendente: }\SpecialCharTok{\{}\BuiltInTok{abs}\NormalTok{(real\_b }\OperatorTok{{-}}\NormalTok{ suma\_desc\_b)}\SpecialCharTok{\}}\SpecialStringTok{"}\NormalTok{)}
\end{Highlighting}
\end{Shaded}

\begin{verbatim}
Valores truncados (1/i³):
[1.0, 0.125, 0.037, 0.0156, 0.008, 0.00462, 0.00291, 0.00195, 0.00137, 0.001]

Suma ascendente (1 a 10)
Suma parcial: 1.0
Suma parcial: 1.12
Suma parcial: 1.15
Suma parcial: 1.16
Suma parcial: 1.16
Suma parcial: 1.16
Suma parcial: 1.16
Suma parcial: 1.16
Suma parcial: 1.16
Suma parcial: 1.16
Suma ascendente = 1.16

Suma descendente (10 a 1)
Suma parcial: 0.001
Suma parcial: 0.00236
Suma parcial: 0.0043
Suma parcial: 0.0072
Suma parcial: 0.0118
Suma parcial: 0.0197
Suma parcial: 0.0353
Suma parcial: 0.0723
Suma parcial: 0.197
Suma parcial: 1.19
Suma descendente = 1.19

Valor real (doble precisión): 1.197532

Errores absolutos:
Ascendente: 0.037531985674193136
Descendente: 0.007531985674193109
\end{verbatim}

Se puede confirmar que la suma ascendente es la más precisa, porque los
términos pequeños no se pierden por el corte o truncamiento. La
diferencia entre las sumas muestra cómo los errores se van acumulando
según el orden de las operaciones.

\begin{enumerate}
\def\labelenumi{\arabic{enumi}.}
\setcounter{enumi}{1}
\item
  La serie de Maclaurin para la función arcotangente converge para −1
  \textless{} 𝑥 ≤ 1 y está dada por

  \(arctan x = \lim_{n \to \infty} P_n(x) = lim_{n \to \infty} \sum_{i=1}^{n} (-1)^{i+1} \frac{x^{2i-1}}{2i-1}\)
\end{enumerate}

\begin{enumerate}
\def\labelenumi{\alph{enumi}.}
\tightlist
\item
  Utilice el hecho de que tan 𝜋⁄4 = 1 para determinar el número n de
  términos de la serie que se necesita sumar para garantizar que
  \(|4𝑃𝑛(1) − 𝜋| < 10^{−3}\)
\end{enumerate}

\begin{Shaded}
\begin{Highlighting}[]
\NormalTok{x }\OperatorTok{=} \DecValTok{1}
\NormalTok{tol }\OperatorTok{=} \FloatTok{1e{-}3}
\NormalTok{n }\OperatorTok{=} \DecValTok{1}

\ControlFlowTok{while} \VariableTok{True}\NormalTok{:}
\NormalTok{    Pn }\OperatorTok{=} \BuiltInTok{sum}\NormalTok{(((}\OperatorTok{{-}}\DecValTok{1}\NormalTok{)}\OperatorTok{**}\NormalTok{(i}\OperatorTok{+}\DecValTok{1}\NormalTok{)) }\OperatorTok{*}\NormalTok{ x}\OperatorTok{**}\NormalTok{(}\DecValTok{2}\OperatorTok{*}\NormalTok{i}\OperatorTok{{-}}\DecValTok{1}\NormalTok{)}\OperatorTok{/}\NormalTok{(}\DecValTok{2}\OperatorTok{*}\NormalTok{i}\OperatorTok{{-}}\DecValTok{1}\NormalTok{) }\ControlFlowTok{for}\NormalTok{ i }\KeywordTok{in} \BuiltInTok{range}\NormalTok{(}\DecValTok{1}\NormalTok{,n}\OperatorTok{+}\DecValTok{1}\NormalTok{))}
\NormalTok{    error }\OperatorTok{=} \BuiltInTok{abs}\NormalTok{(}\DecValTok{4}\OperatorTok{*}\NormalTok{Pn }\OperatorTok{{-}} \FloatTok{3.141592653589793}\NormalTok{)  }\CommentTok{\# 𝜋 real}
    \ControlFlowTok{if}\NormalTok{ error }\OperatorTok{\textless{}}\NormalTok{ tol: }\ControlFlowTok{break}
\NormalTok{    n }\OperatorTok{+=} \DecValTok{1}

\BuiltInTok{print}\NormalTok{(}\StringTok{"n para |4Pn(1){-}𝜋|\textless{}1e{-}3:"}\NormalTok{, n)}
\BuiltInTok{print}\NormalTok{(}\StringTok{"Aproximación de 𝜋:"}\NormalTok{, }\DecValTok{4}\OperatorTok{*}\NormalTok{Pn)}
\end{Highlighting}
\end{Shaded}

\begin{verbatim}
n para |4Pn(1)-𝜋|<1e-3: 1000
Aproximación de 𝜋: 3.140592653839794
\end{verbatim}

\begin{enumerate}
\def\labelenumi{\alph{enumi}.}
\setcounter{enumi}{1}
\tightlist
\item
  El lenguaje de programación C++ requiere que el valor de 𝜋 se
  encuentre dentro de \(10^{10}\). ¿Cuántos términos de la serie se
  necesitarían sumar para obtener este grado de precisión?
\end{enumerate}

La serie de Maclaurin para \((\arctan(x))\) converge muy lentamente
cuando \((x = 1)\).\\
- Para una precisión de \((10^{-3})\), solo se necesitan unos pocos
términos.\\
- Para una precisión de \((10^{-10})\), se requieren miles de términos,
debido a la lentitud de convergencia en el extremo \((x = 1)\).

\begin{enumerate}
\def\labelenumi{\arabic{enumi}.}
\setcounter{enumi}{2}
\tightlist
\item
  Otra fórmula para calcular 𝜋 se puede deducir a partir de la identidad
  \$\frac{𝜋}{4} = 4 arctan \frac{1}{5} - arctan \frac{1}{239} \$.
  Determine el número de términos que se deben sumar para garantizar una
  aproximación 𝜋 dentro de \(10^{-3}\).
\end{enumerate}

\begin{Shaded}
\begin{Highlighting}[]
\ImportTok{import}\NormalTok{ math}

\NormalTok{x1, x2 }\OperatorTok{=} \DecValTok{1}\OperatorTok{/}\DecValTok{5}\NormalTok{, }\DecValTok{1}\OperatorTok{/}\DecValTok{239}
\NormalTok{tol }\OperatorTok{=} \FloatTok{1e{-}3}
\NormalTok{n }\OperatorTok{=} \DecValTok{1}

\ControlFlowTok{while} \VariableTok{True}\NormalTok{:}
\NormalTok{    error }\OperatorTok{=} \DecValTok{16}\OperatorTok{*}\NormalTok{x1}\OperatorTok{**}\NormalTok{(}\DecValTok{2}\OperatorTok{*}\NormalTok{n}\OperatorTok{+}\DecValTok{1}\NormalTok{)}\OperatorTok{/}\NormalTok{(}\DecValTok{2}\OperatorTok{*}\NormalTok{n}\OperatorTok{+}\DecValTok{1}\NormalTok{) }\OperatorTok{+} \DecValTok{4}\OperatorTok{*}\NormalTok{x2}\OperatorTok{**}\NormalTok{(}\DecValTok{2}\OperatorTok{*}\NormalTok{n}\OperatorTok{+}\DecValTok{1}\NormalTok{)}\OperatorTok{/}\NormalTok{(}\DecValTok{2}\OperatorTok{*}\NormalTok{n}\OperatorTok{+}\DecValTok{1}\NormalTok{)}
    \ControlFlowTok{if}\NormalTok{ error }\OperatorTok{\textless{}}\NormalTok{ tol: }\ControlFlowTok{break}
\NormalTok{    n }\OperatorTok{+=} \DecValTok{1}

\NormalTok{Pn }\OperatorTok{=} \KeywordTok{lambda}\NormalTok{ x: }\BuiltInTok{sum}\NormalTok{(((}\OperatorTok{{-}}\DecValTok{1}\NormalTok{)}\OperatorTok{**}\NormalTok{(i}\OperatorTok{+}\DecValTok{1}\NormalTok{)) }\OperatorTok{*}\NormalTok{ x}\OperatorTok{**}\NormalTok{(}\DecValTok{2}\OperatorTok{*}\NormalTok{i}\OperatorTok{{-}}\DecValTok{1}\NormalTok{)}\OperatorTok{/}\NormalTok{(}\DecValTok{2}\OperatorTok{*}\NormalTok{i}\OperatorTok{{-}}\DecValTok{1}\NormalTok{) }\ControlFlowTok{for}\NormalTok{ i }\KeywordTok{in} \BuiltInTok{range}\NormalTok{(}\DecValTok{1}\NormalTok{,n}\OperatorTok{+}\DecValTok{1}\NormalTok{))}
\NormalTok{pi\_approx }\OperatorTok{=} \DecValTok{4}\OperatorTok{*}\NormalTok{(}\DecValTok{4}\OperatorTok{*}\NormalTok{Pn(x1) }\OperatorTok{{-}}\NormalTok{ Pn(x2))}
\BuiltInTok{print}\NormalTok{(}\StringTok{"Términos necesarios:"}\NormalTok{, n)}
\BuiltInTok{print}\NormalTok{(}\StringTok{"π aproximado:"}\NormalTok{, pi\_approx)}
\BuiltInTok{print}\NormalTok{(}\StringTok{"Error absoluto:"}\NormalTok{, }\BuiltInTok{abs}\NormalTok{(math.pi }\OperatorTok{{-}}\NormalTok{ pi\_approx))}
\end{Highlighting}
\end{Shaded}

\begin{verbatim}
Términos necesarios: 3
π aproximado: 3.1416210293250346
Error absoluto: 2.837573524150372e-05
\end{verbatim}

\begin{itemize}
\tightlist
\item
  Se puede observar el valor mínimo de 𝑛
\item
  La fórmula de Machin converge muy rápido debido al término (1/239)
\item
  El 𝑛 por la cota es pequeño.
\end{itemize}

\begin{enumerate}
\def\labelenumi{\arabic{enumi}.}
\setcounter{enumi}{3}
\tightlist
\item
  Compare los siguientes tres algoritmos. ¿Cuándo es correcto el
  algoritmo de la parte 1a?
\end{enumerate}

\begin{center}
\includegraphics[width=5cm,height=10cm]{Algoritmos.png}
\end{center}

\begin{Shaded}
\begin{Highlighting}[]
\NormalTok{valores }\OperatorTok{=}\NormalTok{ [}\DecValTok{2}\NormalTok{, }\DecValTok{3}\NormalTok{, }\DecValTok{4}\NormalTok{]}

\CommentTok{\# Algoritmo a}
\NormalTok{p\_a }\OperatorTok{=} \DecValTok{0}
\ControlFlowTok{for}\NormalTok{ v }\KeywordTok{in}\NormalTok{ valores: p\_a }\OperatorTok{*=}\NormalTok{ v}

\CommentTok{\# Algoritmo b}
\NormalTok{p\_b }\OperatorTok{=} \DecValTok{1}
\ControlFlowTok{for}\NormalTok{ v }\KeywordTok{in}\NormalTok{ valores: p\_b }\OperatorTok{*=}\NormalTok{ v}

\CommentTok{\# Algoritmo c}
\NormalTok{p\_c }\OperatorTok{=} \DecValTok{1}
\ControlFlowTok{for}\NormalTok{ v }\KeywordTok{in}\NormalTok{ valores:}
    \ControlFlowTok{if}\NormalTok{ v }\OperatorTok{==} \DecValTok{0}\NormalTok{: p\_c }\OperatorTok{=} \DecValTok{0}\OperatorTok{;} \ControlFlowTok{break}
\NormalTok{    p\_c }\OperatorTok{*=}\NormalTok{ v}

\BuiltInTok{print}\NormalTok{(}\StringTok{"Producto a:"}\NormalTok{, p\_a)}
\BuiltInTok{print}\NormalTok{(}\StringTok{"Producto b:"}\NormalTok{, p\_b)}
\BuiltInTok{print}\NormalTok{(}\StringTok{"Producto c:"}\NormalTok{, p\_c)}
\end{Highlighting}
\end{Shaded}

\begin{verbatim}
Producto a: 0
Producto b: 24
Producto c: 24
\end{verbatim}

\begin{itemize}
\tightlist
\item
  El \textbf{algoritmo (a)} es correcto cuando \textbf{n = 0},ya que al
  iniciar con \texttt{PRODUCT\ =\ 0}, cualquier multiplicación posterior
  dará 0.
\item
  El \textbf{algoritmo (b)} sirve para calcular el producto de (n)
  números.\\
\item
  El \textbf{algoritmo (c)} es más eficiente si hay ceros en la lista,
  termina de inmediato en ese caso.
\end{itemize}

\subsection{DISCUSIONES}\label{discusiones}

\begin{enumerate}
\def\labelenumi{\arabic{enumi}.}
\tightlist
\item
  Escriba un algoritmo para sumar la serie finita \$
  \sum\_\{i=1\}\^{}\{n\} x\_j \$ en orden inverso.
\end{enumerate}

\begin{Shaded}
\begin{Highlighting}[]
\KeywordTok{def}\NormalTok{ suma\_inversa(x):}
\NormalTok{    S }\OperatorTok{=} \DecValTok{0}
    \ControlFlowTok{for}\NormalTok{ xi }\KeywordTok{in} \BuiltInTok{reversed}\NormalTok{(x):}
\NormalTok{        S }\OperatorTok{+=}\NormalTok{ xi}
    \ControlFlowTok{return}\NormalTok{ S}

\CommentTok{\# Ejemplo}
\NormalTok{valores }\OperatorTok{=}\NormalTok{ [}\DecValTok{1}\NormalTok{, }\DecValTok{2}\NormalTok{, }\DecValTok{3}\NormalTok{, }\DecValTok{4}\NormalTok{, }\DecValTok{5}\NormalTok{]}
\BuiltInTok{print}\NormalTok{(}\StringTok{"Suma de orden inverso:"}\NormalTok{, suma\_inversa(valores))}
\end{Highlighting}
\end{Shaded}

\begin{verbatim}
Suma de orden inverso: 15
\end{verbatim}

Sumar los elementos de una lista desde el último hasta el primero y
permite precisión en las sumas

\begin{enumerate}
\def\labelenumi{\arabic{enumi}.}
\setcounter{enumi}{1}
\tightlist
\item
  Las ecuaciones (1.2) y (1.3) en la sección 1.2 proporcionan formas
  alternativas para las raíces \$𝑥\_1 \$ y \$ 𝑥\_2\$ de
  \(ax^2 + bx +c = 0\). Construya un algoritmo con entrada 𝑎, 𝑏, 𝑐 c y
  salida 𝑥1, 𝑥2 que calcule las raíces \$𝑥\_1 \$ y \$ 𝑥\_2\$ (que pueden
  ser iguales con conjugados complejos) mediante la mejor fórmula para
  cada raíz.
\end{enumerate}

\begin{Shaded}
\begin{Highlighting}[]
\ImportTok{import}\NormalTok{ cmath  }

\KeywordTok{def}\NormalTok{ raices\_cuadratica(a, b, c):}
    \CommentTok{"""Calcula raíces de ax\^{}2 + bx + c = 0"""}
\NormalTok{    discriminante }\OperatorTok{=}\NormalTok{ cmath.sqrt(b}\OperatorTok{**}\DecValTok{2} \OperatorTok{{-}} \DecValTok{4}\OperatorTok{*}\NormalTok{a}\OperatorTok{*}\NormalTok{c)}
    
    \ControlFlowTok{if}\NormalTok{ b }\OperatorTok{\textgreater{}=} \DecValTok{0}\NormalTok{:}
\NormalTok{        x1 }\OperatorTok{=}\NormalTok{ (}\OperatorTok{{-}}\NormalTok{b }\OperatorTok{{-}}\NormalTok{ discriminante) }\OperatorTok{/}\NormalTok{ (}\DecValTok{2}\OperatorTok{*}\NormalTok{a)}
\NormalTok{        x2 }\OperatorTok{=}\NormalTok{ (}\DecValTok{2}\OperatorTok{*}\NormalTok{c) }\OperatorTok{/}\NormalTok{ (}\OperatorTok{{-}}\NormalTok{b }\OperatorTok{{-}}\NormalTok{ discriminante)}
    \ControlFlowTok{else}\NormalTok{:}
\NormalTok{        x1 }\OperatorTok{=}\NormalTok{ (}\OperatorTok{{-}}\NormalTok{b }\OperatorTok{+}\NormalTok{ discriminante) }\OperatorTok{/}\NormalTok{ (}\DecValTok{2}\OperatorTok{*}\NormalTok{a)}
\NormalTok{        x2 }\OperatorTok{=}\NormalTok{ (}\DecValTok{2}\OperatorTok{*}\NormalTok{c) }\OperatorTok{/}\NormalTok{ (}\OperatorTok{{-}}\NormalTok{b }\OperatorTok{+}\NormalTok{ discriminante)}
    
    \ControlFlowTok{return}\NormalTok{ x1, x2}

\CommentTok{\# Ejemplo}
\NormalTok{a, b, c }\OperatorTok{=} \DecValTok{1}\NormalTok{, }\OperatorTok{{-}}\DecValTok{3}\NormalTok{, }\DecValTok{2}
\NormalTok{x1, x2 }\OperatorTok{=}\NormalTok{ raices\_cuadratica(a, b, c)}
\BuiltInTok{print}\NormalTok{(}\SpecialStringTok{f"Raíces: x1 = }\SpecialCharTok{\{}\NormalTok{x1}\SpecialCharTok{\}}\SpecialStringTok{, x2 = }\SpecialCharTok{\{}\NormalTok{x2}\SpecialCharTok{\}}\SpecialStringTok{"}\NormalTok{)}
\end{Highlighting}
\end{Shaded}

\begin{verbatim}
Raíces: x1 = (2+0j), x2 = (1+0j)
\end{verbatim}

\begin{enumerate}
\def\labelenumi{\arabic{enumi}.}
\setcounter{enumi}{2}
\tightlist
\item
  Suponga que
\end{enumerate}

\[ \frac{1-2x}{1-x+x^2} + \frac{2x-4x^3}{1-x^{2}+x^{3}} + \frac{4x^3-8x^7}{1-x^4+x^{8} }  + ... = \frac{1+2x}{1+x+x^{2}}  \]

para 𝑥 \textless{} 1 y si 𝑥 = 0.25. Escriba y ejecute un algoritmo que
determine el número de términos necesarios en el lado izquierdo de la
ecuación de tal forma que el lado izquierdo difiera del lado derecho en
menos de \(10^{-6}\)

\begin{Shaded}
\begin{Highlighting}[]
\NormalTok{x }\OperatorTok{=} \FloatTok{0.25}
\NormalTok{lado\_derecho }\OperatorTok{=} \DecValTok{1} \OperatorTok{+} \DecValTok{2}\OperatorTok{*}\NormalTok{x }\OperatorTok{+}\NormalTok{ x}\OperatorTok{**}\DecValTok{2}
\NormalTok{S }\OperatorTok{=} \DecValTok{0}
\NormalTok{n }\OperatorTok{=} \DecValTok{0}
\NormalTok{tol }\OperatorTok{=} \FloatTok{1e{-}6}
\NormalTok{terminos }\OperatorTok{=}\NormalTok{ [}\DecValTok{1}\NormalTok{, }\DecValTok{2}\OperatorTok{*}\NormalTok{x, x}\OperatorTok{**}\DecValTok{2}\NormalTok{]}

\ControlFlowTok{for}\NormalTok{ t }\KeywordTok{in}\NormalTok{ terminos:}
\NormalTok{    S }\OperatorTok{+=}\NormalTok{ t}
\NormalTok{    n }\OperatorTok{+=} \DecValTok{1}
    \ControlFlowTok{if} \BuiltInTok{abs}\NormalTok{(S }\OperatorTok{{-}}\NormalTok{ lado\_derecho) }\OperatorTok{\textless{}}\NormalTok{ tol:}
        \ControlFlowTok{break}

\BuiltInTok{print}\NormalTok{(}\StringTok{"Términos necesarios:"}\NormalTok{, n)}
\BuiltInTok{print}\NormalTok{(}\StringTok{"Suma lado izquierdo:"}\NormalTok{, S)}
\BuiltInTok{print}\NormalTok{(}\StringTok{"Lado derecho:"}\NormalTok{, lado\_derecho)}
\BuiltInTok{print}\NormalTok{(}\StringTok{"Diferencia:"}\NormalTok{, }\BuiltInTok{abs}\NormalTok{(S }\OperatorTok{{-}}\NormalTok{ lado\_derecho))}
\end{Highlighting}
\end{Shaded}

\begin{verbatim}
Términos necesarios: 3
Suma lado izquierdo: 1.5625
Lado derecho: 1.5625
Diferencia: 0.0
\end{verbatim}

\subsection{Link del repositorio
GitHub}\label{link-del-repositorio-github}

\href{https://github.com/TamyBenavidez/Tarea03.git}{github\_TamyBenavidez},
Tarea N°3




\end{document}
